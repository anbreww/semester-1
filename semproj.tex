%%%%%%%%%%%%%%%%%%%%%%%%%%%%%%%%%%%%%%%%%%%%%%%%%%%%%%%%%%%%%%%%
%%                   Semester project 1	  	              %%
%%              Smart home energy monitoring   		      %%
%%                       Andrew Watson                        %%
%%                           MT-MA1                           %%
%%%%%%%%%%%%%%%%%%%%%%%%%%%%%%%%%%%%%%%%%%%%%%%%%%%%%%%%%%%%%%%%

\input{includes/preamble}
% duplicating graphicx here because of vim-latex autodetect
\usepackage[pdftex]{graphicx} 
\begin{document}
\begin{titlepage}
\nocite{*}      % to make sure bibliography appears in the correct order
  \begin{center}
     
     
    % Upper part of the page
    \includegraphics[width=4cm]{logo_epfl}\\[1.5cm]
     
    \textsc{\LARGE Microengineering }\\[1.0cm]

    \textsc{\Large ELab - Semester Project}\\[0.1cm]

    \vfill 
     
    % Title
    \HRule \\[0.7cm]
    { \huge \bfseries Smart home wireless sensing}\\[0.4cm]

    \includegraphics[width=0.6\textwidth]{title_pic} 
     
    \HRule \\[2.0cm]
     
    %% Author and supervisor
    \begin{minipage}{0.4\textwidth}
      \begin{flushleft} \large
        \emph{Author:} \\
        Andrew \textsc{Watson}\\[0.8cm]

        Microengineering\\
        Master - Semester 1\\[0.5cm]
      \end{flushleft}
    \end{minipage}
    \begin{minipage}{0.4\textwidth}
      \begin{flushright} \large
        \emph{Supervisor:} \\
        Maher \textsc{Kayal}\\[0.5cm]

        \emph{Assistants:} \\
        Fabrizio \textsc{Lo Conte}\\
        Laurent \textsc{Fabre}\\[0.5cm]
      \end{flushright}
    \end{minipage} \\[2cm]
     
    \vfill
     
    % Bottom of the page
    {\large \today}
     
  \end{center}

\end{titlepage}
%\maketitle

\newpage{}

%\fancyhead{}
\fancyfoot{}
\lhead{}
\cfoot{\thepage}        % numéro de page..
\lfoot{Semester Project}
%\rfoot{\today}
\rfoot{\today} %% TODO : fix the date

%\begin{abstract}
%\end{abstract}


%\renewcommand\contentsname{Plan}  % Rename ``Table des Matières''
\tableofcontents{}

\newpage

%%%% BEGIN : LSTLISTINGS CONFIG %%%%%%
%%%% TODO : MOVE TO SEPARATE FILE ONCE FINISHED %%%%%
%% see http://www.jorgemarsal.com/blog/2009/06/08/source-code-snippets-in-latex/
\lstset{language=C}
%\definecolor{lightgrey}{RGB}{200,200,200}
\definecolor{grey92}{gray}{0.92}
\definecolor{grey75}{gray}{0.75}
\definecolor{grey45}{gray}{0.45}

% todo : add macros for ATXMEGA, AVR?

\lstdefinestyle{console}
{
  numbers=none,
  basicstyle=\bf\ttfamily,
  backgroundcolor=\color{grey92},
  frame=lrtb,
  framerule=0.5pt,
  linewidth=\textwidth,
}
\lstdefinestyle{avr-c}
{
  style=console
}

\lstset{
  style=console
}

%%%%%%% END : LSTLISTINGS CONFIG %%%%%%%%


\section*{Introduction}
\addcontentsline{toc}{section}{Introduction}
\markboth{Introduction}{\MakeUppercase{Introduction}}
% TODO : introduction
This is the introduction for my semester project

Smart homes, energy monitoring and management, etc. Talk about it here or in
``context''?

\section{Context and previous work}
% todo : context
Here, I talk about Thierry's project, and a few words about the other projects,
and how they fit together as a system.

Key points
  - Have a global overview of who is using what when.
  - Be able to take into account many things (movement, temperature in various
  rooms, ambient light to adjust artificial lighting)
  - Reduce power consumption globally by
    + making people more aware of their energy consumption
    + automating what can be automated

\section{Overview}
% todo : overview
Describe how my system works, and why it's useful
This project aims to extend the Powerline network with the addition of wireless
nodes. At the moment, the existing system can control appliances, measure power
consumption, and be controlled by various interfaces. However, there is not yet
a way to connect light switches in places where there is no available power, or
to place sensors at various places, including the garden, for instance.

This project is a step in that direction. The goal for this semester project is
not necessarily to produce a final design for a wireless sensor network, but
rather to produce a working demonstration to show how the technology could be
used, as well as to explore some possibilities in the reduction of power usage.
The wireless nodes that comprise the network must operate on a small power
source for as long a time as possible.

These were the main considerations when researching solutions for the various
functional blocks of the project. When possible, I favoured ease of use and
quick development over slight savings. Most of these could be made up for in a
later stage, if required.

\section{Solution}
% todo : solutions, and find an english title
List some options that were considered and discarded, and why.

\subsection{Power supply}
% todo : power supply
Why NiMH? Why not LiPO? Why not use a DC-DC converter?

\subsection{Wireless}
% todo : wireless
Why ZigBee? Within ZigBee solutions, why XBees?
Several wireless communication solutions were considered during the design
phase. 

% TODO : turn this into real text
- make a table to compare wireless devices

main advantages of XBee : 
- my previous experience with them (no need to learn a
  whole new protocol. Quite nice since I already have to do the hardware design
  and learn a whole new microcontroller architecture)
- No antenna circuitry to design
- Easy to re-use in future projects (only need a specific footprint)
- No proprietary stack -> can use any type of microcontroller

\subsection{Microcontroller}
Why go for an ATXMEGA?
Explain the reason for getting a 64-pin instead of the cheaper XMEGAxxA4.
Don't forget to mention : in the hardware design, only I/Os that are also
available on the 44 pin version were used. Also, because of the good modularity
of the XMEGA family, the code can be ported without any changes from one device
to another.


\section{Network topology}

Zigbee is designed to work well in a mesh configuration. However, this requires
some ``repeater'' nodes (routers) to be constantly on to relay transmissions
from the end nodes. Since the battery-powered devices cannot be expected to run
continuously, the network will function in a star topology.

In this configuration, the gateway device is powered from the grid, and the
wireless sensor nodes wake up at a given interval and transmit their data to the
gateway. The sensors manage their sleep themselves, and it is not necessary to
send them commands. This option has been disabled to save power, but it would be
possible to wake a module up periodically and listen for incoming transmissions.

In this setup, the router and coordinator automatically buffer packets until the
destination node wakes up.

\section{Hardware design}

Go over hardware in blocks

Explain the difference between receiver and emitters

Show details of sensor expansion port

\subsection{Base board}

Each sensor node consists of a basic board and a sensor add-on. The basic board
provides standard features common to all the nodes, such as network
connectivity, power management, and processing.

\subsubsection{Overview}

The main board is made up of several elements. At its core is an ATXMEGA. This
microcontroller was chosen mainly for its scalability and low power
requirements. The microcontroller communicates through a serial link with the
XBee, its radio transmitter.

The sensor nodes have a few on-board sensors which are available on all versions
of the board. These include an ambient light sensor, which is pointed towards
the front of the board, some measurements of supply voltages, as well as
temperature of the microcontroller. As the power usage is meant to be very low,
it was assumed that the temperature readings would not suffer very much from
the activity of the chip itself.

Power is supplied mainly by batteries which are placed directly on the board,
and can be recharged from an external interface. A power switch can be toggled
by the user to turn the node on and off.

\subsubsection{Microcontroller}

The current version of the board uses a 64-pin microcontroller, because it
was impossible to source a 44-pin version at design time. The schematics and
code were designed with a 44 pin microcontroller in mind, and it would be
trivial to adapt the design to the smaller version to reduce cost, and use less
space on the board.

\begin{figure}[htpb]
  \begin{center}
    \includegraphics[width=0.9\textwidth]{blocks/wm_microcontroller}
  \end{center}
  \caption{ATXMEGA Microcontroller}
  \label{fig:microcontroller}
\end{figure}


\subsubsection{Power supply}
% TODO : insert picture of power supply design

The nodes can have various power sources. They were designed to function on two
AA batteries, which provide voltages between 2 and \unit[3.2]{V}. All components on
the board were chosen to operate in this voltage range.

To reduce the size requirements, the batteries are placed directly on the board,
on either side of the the radio transmitter.

On the left side of the board, a 4-pin JST header provides a connection to an
external charger (or solar power module). The power lines can be disconnected by
the main controller, and two communication lines are available for information
exchange between the sensor node and its power source. These lines are connected
to the ATXMEGA's USART lines in case serial communication is desired.

\begin{figure}[h]
  \begin{center}
    \includegraphics[width=0.8\textwidth]{blocks/wm_solar_charging}
  \end{center}
  \caption{External charger connection}
  \label{fig:solar-charger}
\end{figure}

In addition to this, the input voltage can be measured with a voltage divider,
and disconnected if it is not needed. The ATXMEGA has a built-in module for
measuring its own supply voltage, using a 10x voltage divider with a \unit[1]{V}
reference.

\subsubsection{Debug}

The nodes were designed to function the same way if the microcontroller was
replaced by a 44-pin version. This leaves one port unused on the ATMEGA64A3,
which was thus used as a debug port. Five pins are available on the right side
of the device to display status on an LED or to send debug data through USARTF0.

\subsubsection{Discarded features}

Several features originally included in the design or enumerated during the
design phase were discarded. I will go over some of these and explain the
reasons behind these choices.

% TODO : discuss board layout?

\subsection{Sensor boards}

The main feature of the base board is its ``expansion port''. In order for the
nodes to share as much hardware as possible, they were all designed with the
same basic peripherals. All the additional sensors and interfaces specific to
different types of nodes were to be added on an expansion board which could be
exchanged between devices. As such, it is possible to change the structure of a
network simply by replacing a very small part of the hardware. As the plug-in
boards are very small, they are also very cheap and easily replaceable.

Although none of the current sensor boards uses all the features of the
expansion port, is was designed to provide any functionality that one may
require. The expansion port uses a two-way, 20-pin connector with a \unit[2]{mm}
pitch.

\begin{figure}[hp]
  \begin{center}
    \includegraphics[width=0.7\textwidth]{blocks/wm_sensor_io}
  \end{center}
  \caption{Sensor expansion port}
  \label{fig:sensor-io}
\end{figure}

As can be seen on the schematic, the expansion port provides the following
connectivity:
\begin{description}
  \item [Switchable power supply] The port provides two supply voltages, one of
    which can be turned on and off from the microcontroller
  \item[ADC] Several pins are connected to the analog to digital converter.
  \item[DAC] Both digital to analog pins are available.
  \item[AC] Two pins can be routed to the analog comparator module.
  \item[USART] The USART module provides serial communication.
  \item[\IIC{}] An \IIC{} bus is available to communicate with sensors.
  \item[SPI] data lines for master or slave functionality.
\end{description}

Due to the layout of the board, the sensor add-ons have to be very small. In
order to reduce the vertical space needed by the sensors nodes, the batteries
and radio transmitters were placed on the same face of the board as the add-on
connector. Therefore, the space left is only approximately \unit[25x25]{mm$^2$}.
% todo : find exact measurements
This is sufficient however for most basic applications. If more space is needed,
it is assumed that the designer will route the necessary connections on the
add-on board, then connect any other sensors off-board using connector cables.


\subsubsection{Motion sensor}

% TODO : detect movement, low consumption but must be constantly powered (use
% analog comparator), needs min 5v

As its name indicates, the purpose of this board is to detect movement. This
task is achieved by some commercially available PIR sensors. These are cheap and
easy to use, but the downside is that they require a \unit[5]{V} supply voltage
to function.

The hardware design of this board is straightforward : a charge pump converter
provides the supply voltage to the sensor(s), whose inputs are connected to two
microcontroller input pins, where they can trigger interrupts. One sensor board
is equipped with two connectors, to potentially drive two seperate PIR modules
from one wireless node.

% TODO : check name of regulator, add part number

% TODO : mention which annex contains motion sensor schematic
\begin{figure}[!h]
  \begin{center}
    \includegraphics[width=1\textwidth]{flowcharts/motion-flowchart-colour}
  \end{center}
  \caption{Motion sensor node operation}
  \label{fig:motion-flowchart}
\end{figure}

Although they require very little power to function, the PIR modules need a
permanent supply voltage in order to function. Therefore, the power supply must
be constantly enabled. Fortunately, since these modules produce a logical
output, the radio and microcontroller can remain in deep sleep mode until the
interrupt is triggered. For optimal performance though, it would be best to use
a sensor which can operate at the same voltages as the base module, therefore
improving the overall efficiency by removing the charge pump regulator.

% TODO : insert measurements of power consumption

In the current demonstration, the motion sensor node is configured to turn on a
lamp when motion is detected. If no movement is detected for a period of 30
seconds, the light is turned off again.

% TODO : link to website and note article numbers.

\subsubsection{Environmental sensors}

The environment sensing board is comprised of three sensors:

%\begin{enum}
%  \item Barometer
%  \item Temperature sensor
%  \item Capacitive humidity sensor
%\end{enum}


The barometer and temperature were both chosen for their low consumption and
easy communication (both use the I$^{2}$C bus on the expansion port).

The capacitive humidity sensor was chosen for it's price, but as a consequence,
it is very inaccurate and requires calibration.

\subsubsection{Light controller}

The final module is intended to control lights. It include connections for two
momentary switches, as well as two potentiometers which can be turned on and
off. With this module, the microcontroller can be configured to remain in deep
sleep and only awake on a hardware interruption generated by the switches, or to
wake up periodically to sample the potentiometer.

The board itself includes footprints for SMD switches, for testing purposes, but
since the nodes are eventually meant to reside in cases, the buttons and
potentiometers are to be fixed to the enclosure and connected to the add-on
board using cables. This allows a little flexibility in the physical design of
the nodes.

\begin{figure}[!h]
  \begin{center}
    \includegraphics[width=1\textwidth]{flowcharts/dimmer-flowchart-colour}
  \end{center}
  \caption{Dimmer node operation}
  \label{fig:dimmer-flowchart}
\end{figure}

%%%%%%%%%%%%%%%%%%%%%%%%%%%%
% TODO : sampling the potentiometer

% extremities are connected between ground and a microcontroller pin. middle pin
% (find what that's called : cursor?) goes on adc input. pot can be turned on
% and off.



\section{Software considerations}

As all the nodes (gateway and end devices) share the same hardware platform, it
makes sense to share as much as possible of the code base between devices.

The original idea had been to write the same code for all the devices, who could
then automatically detect which role they were to play in the network, and
configure themselves at runtime to fulfil the necessary function. This approach,
though probably feasible in terms of code size, would have required too much
work early on in the hardware design, as well as more code than could probably
be expected for a semester project. It would in my mind be more interesting to
have a very modular system than to hard-code the functionality of various
modules, and have several code bases for different nodes.

With this in mind, the code is organised in the following way:

% todo : insert diagram of code hierarchy

All code shares a common root. The folder \bash{xmega\_libs} contain libraries
that are specific to the XMEGA microcontroller, mostly provided by Atmel.
Although many hardware-level functions and registers were used in the main
program, mainly for lack of time, the main objective would be to abstract all
the hardware-level functionality to these libraries, to ease portability to
another microcontroller platform.

Next we have \bash{project\_libs}, which contains all shared code relevant to
this semester project. In here, we have libraries for communicating with the
XBee radios, sending commands to the Powerline modules, as well as collecting
information from all types of sensor modules.

Finally, in the \bash{project\_libs} folder, the \bash{hardware.h} file contains
all the pin definitions relevant to the layout of the board. All functions
should use the definitions provided in this file rather than hard-code port
names. This way, existing code can be made to work on a different hardware
design simply by replacing the \bash{hardware.h} file.

Lastly, since each type of node behaves differently, it has its own set of
files. The \bash{motion\_sensor} folder contains the behaviour for a motion
sensing node, \bash{dimmer\_board} contains the code for a dimmer node, and so
on. Usually, the \bash{main.c} file is the only file that differs between the
various nodes.

\section{Low-power design}

\section{Cost considerations}

For the sake of ease of development, some sacrifices were made in the area of
cost. For starters, the ATXMEGA is not very well suited to extremely low-cost
designs, as it is quite a lot more expensive than some competing solutions,
especially the PIC24F series and Cortex M3. On the other hand, it overs a very
modern architecture with many features to experiment with, which allows a lot of
freedom for development and experimenting.

% todo : mention choice of xmegaA3 over A4 and stuff
Moreover,

The other main area were cost was overlooked at the benefit of development time
is the wireless interface. Indeed, the average cost of an XBee module is
approximately \$22, whereas it is possible to source modules from Microchip for
instance that cost less than half as much. However, these require a one-time
purchase of an expensive software stack (which is not included), as well as a
more complex development. Therefore, the XBee modules were chosen, for their
seemingly ``plug-and-play'' functionality. If more time could be spend on the
development, it would be reasonable to expect to spend up to three times less on
the radio transmitter modules, while keeping the same functionality and
compatibility with the existing modules.

\section{Test results}

\section{Some examples of code}
\begin{lstlisting}[language=C]
  #define F_CPU 8000000UL
  #define Blah() do_nothing();

  #include <avr/io.h>
  #include "stuff.h"
  int main(void)
  {
    char blue = 0;
    int i = 0;

    for(i=0; i<10; i++)
    {
      // do stuff
      foo();
      bar();
    }
    code_example();
  }
\end{lstlisting}

There's also a way of including \code{code} inline.


\end{document}

