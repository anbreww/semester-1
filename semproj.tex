%%%%%%%%%%%%%%%%%%%%%%%%%%%%%%%%%%%%%%%%%%%%%%%%%%%%%%%%%%%%%%%%
%%                   Semester project 1	  	              %%
%%              Smart home energy monitoring   		      %%
%%                       Andrew Watson                        %%
%%                           MT-MA1                           %%
%%%%%%%%%%%%%%%%%%%%%%%%%%%%%%%%%%%%%%%%%%%%%%%%%%%%%%%%%%%%%%%%

\input{includes/preamble}
\begin{document}
\begin{titlepage}
\nocite{*}      % to make sure bibliography appears in the correct order
  \begin{center}
     
     
    % Upper part of the page
    %\includegraphics[width=4cm]{logo_epfl}\\[0.5cm]
     
    \vfill 
    \textsc{\LARGE Ecole Polytechnique Fédérale de Lausanne }\\[1.0cm]

    { \huge \bfseries Smart home energy monitoring}\\[0.4cm]
    \includegraphics[width=12cm]{images/logo_epfl}\\[0.5cm]

    \vfill

     
    % Bottom of the page
    Andrew \textsc{Watson}\\[0.5cm] 
    Version 0.1.1\\[0.5cm]
    {\large \today} 
     
  \end{center}

\end{titlepage}
%\maketitle

\newpage{}

%\fancyhead{}
\fancyfoot{}
\lhead{}
\cfoot{\thepage}        % numéro de page..
\lfoot{Semester Project}
%\rfoot{\today}
\rfoot{\today} %% TODO : fix the date

%\begin{abstract}
%\end{abstract}

% CUSTOM COMMANDS FOR THIS REPORT
\newcommand{\expr}[1]{\og \emph{{#1}} \fg} % \expr{word} => « Word »
\newcommand{\important}[1]{\textbf{#1}}
\newcommand{\code}[1]{\texttt{#1}}

%\renewcommand\contentsname{Plan}  % Rename ``Table des Matières''
\tableofcontents{}

\newpage

%%%% BEGIN : LSTLISTINGS CONFIG %%%%%%
%%%% TODO : MOVE TO SEPARATE FILE ONCE FINISHED %%%%%
%% see http://www.jorgemarsal.com/blog/2009/06/08/source-code-snippets-in-latex/
\lstset{language=C}
%\definecolor{lightgrey}{RGB}{200,200,200}
\definecolor{grey92}{gray}{0.92}
\definecolor{grey75}{gray}{0.75}
\definecolor{grey45}{gray}{0.45}
\lstdefinestyle{console}
{
  numbers=none,
  basicstyle=\bf\ttfamily,
  backgroundcolor=\color{grey92},
  frame=lrtb,
  framerule=0.5pt,
  linewidth=\textwidth,
}
\lstdefinestyle{avr-c}
{
  style=console
}

\lstset{
  style=console
}

%%%%%%% END : LSTLISTINGS CONFIG %%%%%%%%


\section*{Introduction}
\addcontentsline{toc}{section}{Introduction}
\markboth{Introduction}{\MakeUppercase{Introduction}}
This is the introduction for my semester project

\section{Context and previous work}
Here, I talk about Thierry's project, and a few words about the other projects,
and how they fit together as a system.

\section{Overview}
Describe how my system works, and why it's useful

\section{Recherche de solutions}
List some options that were considered and discarded, and why.

\subsection{Power supply}
Why NiMH? Why not LiPO? Why not use a DC-DC converter?

\subsection{Wireless}
Why ZigBee? Within ZigBee solutions, why XBees?

\subsection{Microcontroller}
Why go for an ATXMEGA?
Explain the reason for getting a 64-pin instead of the cheaper XMEGAxxA4.
Don't forget to mention : in the hardware design, only I/Os that are also
available on the 44 pin version were used. Also, because of the good modularity
of the XMEGA family, the code can be ported without any changes from one device
to another.

\section{Network topology}

\section{Hardware design}

Go over hardware in blocks

Explain the difference between receiver and emitters

Show details of sensor expansion port

\section{Software considerations}

\section{Low-power design}

\section{Test results}

\section{Some examples of code}
\begin{lstlisting}[style=avr-c]
  int main(void)
  {
    code_example();
  }
\end{lstlisting}

There's also a way of including \code{code} inline.

\end{document}
