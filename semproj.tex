%%%%%%%%%%%%%%%%%%%%%%%%%%%%%%%%%%%%%%%%%%%%%%%%%%%%%%%%%%%%%%%%
%%                   Semester project 1	  	              %%
%%              Smart home energy monitoring   		      %%
%%                       Andrew Watson                        %%
%%                           MT-MA1                           %%
%%%%%%%%%%%%%%%%%%%%%%%%%%%%%%%%%%%%%%%%%%%%%%%%%%%%%%%%%%%%%%%%

\input{includes/preamble}
\begin{document}
\begin{titlepage}
\nocite{*}      % to make sure bibliography appears in the correct order
  \begin{center}
     
     
    % Upper part of the page
    \includegraphics[width=4cm]{logo_epfl}\\[1.5cm]
     
    \textsc{\LARGE Microengineering }\\[1.0cm]

    \textsc{\Large ELab - Semester Project}\\[0.1cm]

    \vfill 
     
    % Title
    \HRule \\[0.7cm]
    { \huge \bfseries Smart home wireless sensing}\\[0.4cm]

    \includegraphics[width=0.6\textwidth]{title_pic} 
     
    \HRule \\[2.0cm]
     
    %% Author and supervisor
    \begin{minipage}{0.4\textwidth}
      \begin{flushleft} \large
        \emph{Author:} \\
        Andrew \textsc{Watson}\\[0.8cm]

        Microengineering\\
        Master - Semester 1\\[0.5cm]
      \end{flushleft}
    \end{minipage}
    \begin{minipage}{0.4\textwidth}
      \begin{flushright} \large
        \emph{Supervisor:} \\
        Maher \textsc{Kayal}\\[0.5cm]

        \emph{Assistants:} \\
        Fabrizio \textsc{Lo Conte}\\
        Laurent \textsc{Fabre}\\[0.5cm]
      \end{flushright}
    \end{minipage} \\[2cm]
     
    \vfill
     
    % Bottom of the page
    {\large \today}
     
  \end{center}

\end{titlepage}
%\maketitle

\newpage{}

%\fancyhead{}
\fancyfoot{}
\lhead{}
\cfoot{\thepage}        % numéro de page..
\lfoot{Semester Project}
%\rfoot{\today}
\rfoot{\today} %% TODO : fix the date

%\begin{abstract}
%\end{abstract}


%\renewcommand\contentsname{Plan}  % Rename ``Table des Matières''
\tableofcontents{}

\newpage

%%%% BEGIN : LSTLISTINGS CONFIG %%%%%%
%%%% TODO : MOVE TO SEPARATE FILE ONCE FINISHED %%%%%
%% see http://www.jorgemarsal.com/blog/2009/06/08/source-code-snippets-in-latex/
\lstset{language=C}
%\definecolor{lightgrey}{RGB}{200,200,200}
\definecolor{grey92}{gray}{0.92}
\definecolor{grey75}{gray}{0.75}
\definecolor{grey45}{gray}{0.45}


\lstdefinestyle{console}
{
  numbers=none,
  basicstyle=\bf\ttfamily,
  backgroundcolor=\color{grey92},
  frame=lrtb,
  framerule=0.5pt,
  linewidth=\textwidth,
}
\lstdefinestyle{avr-c}
{
  style=console
}

\lstset{
  style=console
}

%%%%%%% END : LSTLISTINGS CONFIG %%%%%%%%


\section*{Introduction}
\addcontentsline{toc}{section}{Introduction}
\markboth{Introduction}{\MakeUppercase{Introduction}}
This is the introduction for my semester project

\section{Context and previous work}
Here, I talk about Thierry's project, and a few words about the other projects,
and how they fit together as a system.

\section{Overview}
Describe how my system works, and why it's useful

\section{Recherche de solutions}
List some options that were considered and discarded, and why.

\subsection{Power supply}
Why NiMH? Why not LiPO? Why not use a DC-DC converter?

\subsection{Wireless}
Why ZigBee? Within ZigBee solutions, why XBees?

\subsection{Microcontroller}
Why go for an ATXMEGA?
Explain the reason for getting a 64-pin instead of the cheaper XMEGAxxA4.
Don't forget to mention : in the hardware design, only I/Os that are also
available on the 44 pin version were used. Also, because of the good modularity
of the XMEGA family, the code can be ported without any changes from one device
to another.

\section{Network topology}

Zigbee is designed to work well in a mesh configuration. However, this requires
some ``repeater'' nodes (routers) to be constantly on to relay transmissions
from the end nodes. Since the battery-powered devices cannot be expected to run
continuously, the network will function in a star topology.

In this configuration, the gateway device is powered from the grid, and the
wireless sensor nodes wake up at a given interval and transmit their data to the
gateway. The sensors manage their sleep themselves, and it is not necessary to
send them commands. This option has been disabled to save power, but it would be
possible to wake a module up periodically and listen for incoming transmissions.

In this setup, the router and coordinator automatically buffer packets until the
destination node wakes up.
\section{Hardware design}

Go over hardware in blocks

Explain the difference between receiver and emitters

Show details of sensor expansion port

\subsection{Sensor boards}

in its current state, the project has three sensor boards that can be installed
in the expansion port.

\subsubsection{Motion sensor}

% TODO : detect movement, low consumption but must be constantly powered (use
% analog comparator), needs min 5v

\subsubsection{Environmental sensors}

The environment sensing board is comprised of three sensors:

%\begin{enum}
%  \item Barometer
%  \item Temperature sensor
%  \item Capacitive humidity sensor
%\end{enum}


The barometer and temperature were both chosen for their low consumption and
easy communication (both use the I$^{2}$C bus on the expansion port).

The capacitive humidity sensor was chosen for it's price, but as a consequence,
it is very inaccurate and requires calibration.

\subsubsection{Light controller}

The final module is intended to control lights. It include connections for two
momentary switches, as well as two potentiometers which can be turned on and
off. With this module, the microcontroller can be configured to remain in deep
sleep and only awake on a hardware interruption generated by the switches, or to
wake up periodically to sample the potentiometer.

%%%%%%%%%%%%%%%%%%%%%%%%%%%%
% sampling the potentiometer

% extremities are connected between ground and a microcontroller pin. middle pin
% (find what that's called : cursor?) goes on adc input. pot can be turned on
% and off.



\section{Software considerations}

\section{Low-power design}

\section{Test results}

\section{Some examples of code}
\begin{lstlisting}[language=C]
  #define F_CPU 8000000UL
  #define Blah() do_nothing();

  #include <avr/io.h>
  #include "stuff.h"
  int main(void)
  {
    char blue = 0;
    int i = 0;

    for(i=0; i<10; i++)
    {
      // do stuff
      foo();
      bar();
    }
    code_example();
  }
\end{lstlisting}

There's also a way of including \code{code} inline.

\end{document}
