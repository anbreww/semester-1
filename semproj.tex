%%%%%%%%%%%%%%%%%%%%%%%%%%%%%%%%%%%%%%%%%%%%%%%%%%%%%%%%%%%%%%%%
%%                   Semester project 1	  	              %%
%%              Smart home energy monitoring   		      %%
%%                       Andrew Watson                        %%
%%                           MT-MA1                           %%
%%%%%%%%%%%%%%%%%%%%%%%%%%%%%%%%%%%%%%%%%%%%%%%%%%%%%%%%%%%%%%%%

\input{includes/preamble}
% duplicating graphicx here because of vim-latex autodetect
\usepackage[pdftex]{graphicx} 
\usepackage{pdfpages}
\begin{document}
\begin{titlepage}
\nocite{*}      % to make sure bibliography appears in the correct order
  \begin{center}
     
     
    % Upper part of the page
    \includegraphics[width=4cm]{logo_epfl}\\[1.5cm]
     
    \textsc{\LARGE Microengineering }\\[1.0cm]

    \textsc{\Large ELab - Semester Project}\\[0.1cm]

    \vfill 
     
    % Title
    \HRule \\[0.7cm]
    { \huge \bfseries Smart home wireless sensing}\\[0.4cm]

    \includegraphics[width=0.6\textwidth]{sideview} 
     
    \HRule \\[2.0cm]
     
    %% Author and supervisor
    \begin{minipage}{0.4\textwidth}
      \begin{flushleft} \large
        \emph{Author:} \\
        Andrew \textsc{Watson}\\[1.0cm]

        Microengineering\\
        Master - Semester 1\\[0.5cm]
      \end{flushleft}
    \end{minipage}
    \begin{minipage}{0.4\textwidth}
      \begin{flushright} \large
        \emph{Supervisor:} \\
        Maher \textsc{Kayal}\\[0.5cm]

        \emph{Assistants:} \\
        Fabrizio \textsc{Lo Conte}\\
        Laurent \textsc{Fabre}\\[0.5cm]
      \end{flushright}
    \end{minipage} \\[2cm]
     
    \vfill
     
    % Bottom of the page
    {\large \today}
     
  \end{center}

\end{titlepage}
%\maketitle

\newpage{}

%\fancyhead{}
\fancyfoot{}
\lhead{}
\cfoot{\thepage}        % numéro de page..
\lfoot{Semester Project}
%\rfoot{\today}
\rfoot{\today} %% TODO : fix the date

%\begin{abstract}
%\end{abstract}

\setcounter{secnumdepth}{5}

%\renewcommand\contentsname{Plan}  % Rename ``Table des Matières''
\tableofcontents{}

\newpage

%%%% BEGIN : LSTLISTINGS CONFIG %%%%%%
%%%% TODO : MOVE TO SEPARATE FILE ONCE FINISHED %%%%%
%% see http://www.jorgemarsal.com/blog/2009/06/08/source-code-snippets-in-latex/
\lstset{language=C}
%\definecolor{lightgrey}{RGB}{200,200,200}
\definecolor{grey97}{gray}{0.97}
\definecolor{grey92}{gray}{0.92}
\definecolor{grey75}{gray}{0.75}
\definecolor{grey45}{gray}{0.45}

% todo : add macros for ATXMEGA, AVR?

% TODO : add references in bib for application notes, esp AVR1010

\lstdefinestyle{console}
{
  numbers=none,
  %basicstyle=\bf\ttfamily,
  basicstyle=\ttfamily\footnotesize,
  backgroundcolor=\color{grey97},
  frame=lrtb,
  framerule=0.5pt,
  linewidth=\textwidth,
}
\lstdefinestyle{avr-c}
{
  style=console
}

\lstset{
  style=console
}

%%%%%%% END : LSTLISTINGS CONFIG %%%%%%%%


\section*{Introduction}
\addcontentsline{toc}{section}{Introduction}
\markboth{Introduction}{\MakeUppercase{Introduction}}
% TODO : introduction
Smart energy management is one of the core applications of home automation. To
this end, many systems have been developed, but not many are in widespread use.
It is still an emerging business which can benefit from research, and is still
being very actively developed at the moment.

Here are some of the main features that a well integrated home automation system
could provide:
\begin{description}
  \item[Consumption] : indicate where energy is being consumed and by whom, as
    well as extract statistics, which can be linked to other factors, like for
    instance thermostat settings, outdoor temperature, activity in the
    home\ldots
  \item[React] to the environment : turn lights and other appliances on and off
    depending on activity and time of day. It would be interesting for example
    to turn off lights when no human presence is detected, to use power-hungry
    devices during the night to even out power usage, or to adjust the lighting
    to suit the amount of natural light coming in a room. 
\end{description}

% todo : cite some numbers in objectives (like power consumption or how long it
% should last)

% todo : citation needed
The benefits of such a system are twofold : first, it has been demonstrated
\textbf{(citation needed)} that simply providing people with data about their
energy consumption results in a substantial drop in power usage. It can also
help to make statistics and comparisons between devices and households.

Second, as described previously, it is possible to add a level of automation to
the system. Actions such as adjusting the lighting based on incoming sunlight or
the level of activity in a room for example are not common, but could lead to
energy savings. These are both fairly straightforward to implement and could
actually be implemented on the system developed in this semester project.

Another interesting application is remote control : rather than leave heating
and other things on all the time while away, it would be possible to turn them
off just after leaving the house, and turn them on a short time before arriving.
With most current-generation smartphones, it would even be possible to do all
this automatically, based on the user's position as determined by the phone's
internal GPS.


%%%%%%%%%%%%%%%%%%%%%%%%%%%%%%%%%%%%%%%%%%%%%%%%%%%%%%%%%%%%%%%%%%%%%%%%%%%%%%%
\section{Context and previous work}
% TODO : context
\textbf{Here : talk about previous projects, how they fit together}


Several projects in the ELab have been focused around the idea of a Powerline
network : modules that plug into a regular household socket and transmit data
along the power lines. The first project by Thierry Barras \cite{barras2009} was
to develop a set of modules that can communicate together on the power network.

Later projects added a Wi-Fi interface, which allowed the nodes to be controlled
by any internet-capable device, a GSM module, which can accept commands by text
messages, and a touch-screen interface.


%%%%%%%%%%%%%%%%%%%%%%%%%%%%%%%%%%%%%%%%%%%%%%%%%%%%%%%%%%%%%%%%%%%%%%%%%%%%%%%
\section{Overview}
% todo : overview
\textbf{Here : what's missing in the existing system, where my project fits in}

This project aims to extend the Powerline network with the addition of wireless
nodes. At the moment, the existing system can control appliances, measure power
consumption, and be controlled by various interfaces. However, there is not yet
a way to connect light switches in places where there is no available power, or
to place sensors at various places, including the garden, for instance.

This project is a step in that direction. The goal for this semester project is
not necessarily to produce a final design for a wireless sensor network, but
rather to produce a working demonstration to show how the technology could be
used, as well as to explore some possibilities in the reduction of power usage.
The wireless nodes that comprise the network must operate on a small power
source for as long a time as possible.

These were the main considerations when researching solutions for the various
functional blocks of the project. When possible, I favoured ease of use and
quick development over slight savings. Most of these could be made up for in a
later stage, if required.

At the end of this project, a small-scale implementation of the wireless sensor
network will be demonstrated. The layout and general operation of the
demonstration is described in \tref{sec:network-topology}. In its current state,
the wireless network can be extended easily by assembling and programming
additional wireless nodes, using the current hardware design.

At the time of writing, only two powerline modules were functional. At the
moment, each wireless node controls a specific powerline module (i.e. the module
with address \#2, the only one available at the time) If the powerline network is
to be extended in the future, it would be interesting to add functionality to
reconfigure the wireless nodes at runtime, to select different targets to
control.

%%%%%%%%%%%%%%%%%%%%%%%%%%%%%%%%%%%%%%%%%%%%%%%%%%%%%%%%%%%%%%%%%%%%%%%%%%%%%%%
\section{Solutions}
% todo : solutions, and find an english title
List some options that were considered and discarded, and why.


%%%%%%%%%%%%%%%%%%%%%%%%%%%%%%%%%%%%%%%%%%%%%%%%%%%%%%%%%%%%%%%%%%%%%%%%%%%%%%%
\subsection{Power supply}
% todo : power supply
Why NiMH? Why not LiPO? Why not use a DC-DC converter?


%%%%%%%%%%%%%%%%%%%%%%%%%%%%%%%%%%%%%%%%%%%%%%%%%%%%%%%%%%%%%%%%%%%%%%%%%%%%%%%
\subsection{Wireless}

Several wireless communication solutions were considered during the design
phase. We will begin with a brief overview of the solutions that were considered
for this application.

%%%%%%%%%%%%%%%%%%%%%%%%%%%%%%%%%%%%%%%%%%%%%%%%%%%%%%%%%%%%%%%%%%%%%%%%%%%%%%%
\subsubsection{Microchip MRF24J40}

Microchip\furl{http://www.microchip.com} offer a series of integrated solutions
based on the 802.15.4\furl{http://en.wikipedia.org/wiki/IEEE_802.15.4-2006}
hardware standard. They are quite affordable (\$10 for the short range version),
but they require the installation of a wireless stack. The options provided by
Microchip are either \emph{MiWi}, which is a royalty-free, lightweight, and free
stack, or a full ZigBee stack.

Although the MiWi stack seems to fit our application quite well, the fact that
it is proprietary basically guarantees that we cannot ensure interoperability
with devices from other vendors, as well as forcing us to use Microchip's
platform for the foreseeable future. This is clearly not desirable as one of my
main objectives for this project was to ensure that it is scalable and can be
extended to other devices.

The ZigBee stack is an interesting alternative, as it is theoretically
compatible with any other ZigBee-certified device. It was discarded immediately
however as the base cost to install and use the framework is way outside the
scope of this project (over \$900 at the time of writing).

%%%%%%%%%%%%%%%%%%%%%%%%%%%%%%%%%%%%%%%%%%%%%%%%%%%%%%%%%%%%%%%%%%%%%%%%%%%%%%%
\subsubsection{Nordic nRF24L01+}

The nRF24 series by Nordic Semiconductor\furl{http://www.nordicsemi.com/} is a
very interesting and cost-effective solution for small wireless networks. Their
chips are quite inexpensive (approx. \unit[7]{Chf}) and they provide a hardware
design which can be re-used with minimal work. Unfortunately, the system doesn't
appear to scale very well, as one chip can only communicate with six different
neighbours at a time. Beyond that, it requires some workarounds such as
frequency hopping, which may be interesting but would require too much work for
a project of this scope.

%%%%%%%%%%%%%%%%%%%%%%%%%%%%%%%%%%%%%%%%%%%%%%%%%%%%%%%%%%%%%%%%%%%%%%%%%%%%%%%
\subsubsection{Other solutions}

A few other solutions were considered, such as the CC2520 family by Texas
Instruments\furl{http://ti.com}. This and many others were discarded because
they require an antenna to be designed into the PCB. These solutions would be
intresting from a cost perspective, as they are almost half as expensive as
other ready-made solutions, but having to design the RF circuitry would greatly
increase the potential for errors, and require too much design time. 

Finally, it was decided that the solution should be based on the IEEE 802.15.4
standard, provide easy interoperability between device and vendor types, and be
easy to implement, to avoid ``reinventing the wheel'' in terms of radio
transmissions.

%%%%%%%%%%%%%%%%%%%%%%%%%%%%%%%%%%%%%%%%%%%%%%%%%%%%%%%%%%%%%%%%%%%%%%%%%%%%%%%
\subsection{ZigBee : defining reasons}
% TODO : turn this into real text

For most of the reasons mentioned above, the solution which was chosen for this
design were a set of XBee transmitters from Digi
International\furl{http://digi.com}. These modules are (in theory) very easy to
set up and get operational in a short time, and ensured that not too much time
would be wasted on the wireless transmission part of the project.

\textbf{say a few words about ZigBee, interoperability, application profiles
etc}

In addition, they are supposedly fully compliant with the
ZigBee\furl{http://zigbee.org} specification, which should allow the project to
be extended with devices from other vendors without having to replace the
existing hardware. The device footprint is easy to re-use in future projects
(simply place two \unit[2]{mm} headers with decoupling capacitors), without any
need for external antenna circuitry. It is also worth noting that these devices
can be reprogrammed and reconfigured wirelessly, from a program running on a PC.

Finally, I already had some personal experience with these devices, which would
give me a little head start in understanding how packets are assembled and
transmitted.

%%%%%%%%%%%%%%%%%%%%%%%%%%%%%%%%%%%%%%%%%%%%%%%%%%%%%%%%%%%%%%%%%%%%%%%%%%%%%%%
\subsubsection{Afterthoughts}
% TODO : mention stuff that might be better after consideration, ie 6LowPAN

%%%%%%%%%%%%%%%%%%%%%%%%%%%%%%%%%%%%%%%%%%%%%%%%%%%%%%%%%%%%%%%%%%%%%%%%%%%%%%%
\subsection{Current implementation}
For reference, the addresses of the five modules used in this project are listed
in \TAB{tab:addresses}. These addresses may be used to communicate to a specific
device if the short (16-bit) local address is not known. The addresses for known
modules are already contained in the source code for quick re-use.

\begin{table}[h]
  \centering
  \begin{tabular}{l|l|l}
    Address & Node ID & Type \\
    \hline
    0x13A2004064EFC2 & NODE0 & Coordinator \\
    0x13A2004064EFCC & NODE1 & End device \\
    0x13A2004064EFEB & NODE2 & End device \\
    0x13A2004064EFC1 & NODE3 & End device \\
    0x13A2004064EFF1 & NODE4 & Router \\
    %\hline
  \end{tabular}
  \caption{Zigbee device address list}
  \label{tab:addresses}
\end{table}

%%%%%%%%%%%%%%%%%%%%%%%%%%%%%%%%%%%%%%%%%%%%%%%%%%%%%%%%%%%%%%%%%%%%%%%%%%%%%%%
\subsection{Microcontroller}
Why go for an ATXMEGA?
Explain the reason for getting a 64-pin instead of the cheaper XMEGAxxA4.
Don't forget to mention : in the hardware design, only I/Os that are also
available on the 44 pin version were used. Also, because of the good modularity
of the XMEGA family, the code can be ported without any changes from one device
to another.


%%%%%%%%%%%%%%%%%%%%%%%%%%%%%%%%%%%%%%%%%%%%%%%%%%%%%%%%%%%%%%%%%%%%%%%%%%%%%%%
\section{Network topology}
\label{sec:network-topology}

Zigbee is designed to work well in a mesh configuration. However, this requires
some ``repeater'' nodes (routers) to be constantly on to relay transmissions
from the end nodes. Since the battery-powered devices cannot be expected to run
continuously, the network will function in a star topology.

In this configuration, the gateway device is powered from the grid, and the
wireless sensor nodes wake up at a given interval and transmit their data to the
gateway. The nodes do not require any incoming transmissions as they are
configured to transmit regularly. This is the most efficient way of saving
power, but it does limit functionality to a certain extent.

If future applications require it though, it is possible to configure Zigbee end
nodes to wake up regularly and listen for incoming transmissions. If an end node
is asleep when a packet is sent to it, it will be buffered by the nearest router
or coordinator until the node wakes up. It is best to avoid this mode though,
since it reduces battery life, and causes an inherent delayed reaction time.

\Fig{fig:topology-flows} illustrates the topology chosen for the demonstration
which will be presented at the end of this project, and shows which way
information flows. 

\begin{figure}[h]
  \begin{center}
    \includegraphics[width=0.8\textwidth]{flowcharts/topology-flows}
  \end{center}
  \caption{Topology chart illustrating information flows}
  \label{fig:topology-flows}
\end{figure}

This scenario demonstrates a simple network of four nodes. At the center of it,
the Gateway acts as a coordinator, and can receive transmissions from any other
node. Any node which wishes to join the network must connect to the coordinator
and is attributed a 16-bit local address. The network uses a local identifier to
regroup all the nodes, which is reffered to as a PAN ID. The coordinator is
configured to use PAN ID \bash{0x1984}.

Two nodes react to user interaction, and send commands either to dim a lamp or
to turn it on when it detects movement. These commands are sent to the gateway
and forwarded to the powerline network.

A third node takes periodic measurements of its sensor values, but rather than
sending them through the powerline network, they are sent directly to a receiver
connected to a computer by USB. From there, the data can be logged and graphed
from a MATLAB or Python script on the host computer. Additionally, it would be
possible to act on the data and send commands to the powerline network from the
computer, wirelessly. This however is not implemented in the current demo.


%%%%%%%%%%%%%%%%%%%%%%%%%%%%%%%%%%%%%%%%%%%%%%%%%%%%%%%%%%%%%%%%%%%%%%%%%%%%%%%
\section{Hardware design}

Go over hardware in blocks

Explain the difference between receiver and emitters

Show details of sensor expansion port

% TODO : also mention USB debugger!!

Some mistakes were noticed only after fabrication, and some corrections have
been made. Since they are not relevant to the understanding of the project, they
have been placed in the appendix. However, it is highly recommended to read
through this errata before starting to use the wireless modules, and especially
if the design files are to be used again or modified. The list can be found in
Section \ref{sec:design-changes}

%%%%%%%%%%%%%%%%%%%%%%%%%%%%%%%%%%%%%%%%%%%%%%%%%%%%%%%%%%%%%%%%%%%%%%%%%%%%%%%
\subsection{Base board}

Each sensor node consists of a basic board and a sensor add-on. The basic board
provides standard features common to all the nodes, such as network
connectivity, power management, and processing.

%%%%%%%%%%%%%%%%%%%%%%%%%%%%%%%%%%%%%%%%%%%%%%%%%%%%%%%%%%%%%%%%%%%%%%%%%%%%%%%
\subsubsection{Overview}

The main board is made up of several elements. At its core is an ATXMEGA. This
microcontroller was chosen mainly for its scalability and low power
requirements. The microcontroller communicates through a serial link with the
XBee, its radio transmitter.

The sensor nodes have a few on-board sensors which are available on all versions
of the board. These include an ambient light sensor, which is pointed towards
the front of the board, some measurements of supply voltages, as well as
temperature of the microcontroller. As the power usage is meant to be very low,
it was assumed that the temperature readings would not suffer very much from
the activity of the chip itself.

Power is supplied mainly by batteries which are placed directly on the board,
and can be recharged from an external interface. A power switch can be toggled
by the user to turn the node on and off.

%%%%%%%%%%%%%%%%%%%%%%%%%%%%%%%%%%%%%%%%%%%%%%%%%%%%%%%%%%%%%%%%%%%%%%%%%%%%%%%
\subsubsection{Microcontroller}

The current version of the board uses a 64-pin microcontroller, because it
was impossible to source a 44-pin version at design time. The schematics and
code were designed with a 44 pin microcontroller in mind, and it would be
trivial to adapt the design to the smaller version to reduce cost, and use less
space on the board.

\begin{figure}[htpb]
  \begin{center}
    \includegraphics[width=0.9\textwidth]{blocks/wm_microcontroller}
  \end{center}
  \caption{ATXMEGA Microcontroller}
  \label{fig:microcontroller}
\end{figure}


%%%%%%%%%%%%%%%%%%%%%%%%%%%%%%%%%%%%%%%%%%%%%%%%%%%%%%%%%%%%%%%%%%%%%%%%%%%%%%%
\subsubsection{Power supply}
% TODO : insert picture of power supply design

The nodes can have various power sources. They were designed to function on two
AA batteries, which provide voltages between 2 and \unit[3.2]{V}. All components on
the board were chosen to operate in this voltage range.

To reduce the size requirements, the batteries are placed directly on the board,
on either side of the the radio transmitter.

On the left side of the board, a 4-pin JST header provides a connection to an
external charger (or solar power module). The power lines can be disconnected by
the main controller, and two communication lines are available for information
exchange between the sensor node and its power source. These lines are connected
to the ATXMEGA's USART lines in case serial communication is desired.

\begin{figure}[h]
  \begin{center}
    \includegraphics[width=0.8\textwidth]{blocks/wm_solar_charging}
  \end{center}
  \caption{External charger connection}
  \label{fig:solar-charger}
\end{figure}

In addition to this, the input voltage can be measured with a voltage divider,
and disconnected if it is not needed. The ATXMEGA has a built-in module for
measuring its own supply voltage, using a 10x voltage divider with a \unit[1]{V}
reference.

%%%%%%%%%%%%%%%%%%%%%%%%%%%%%%%%%%%%%%%%%%%%%%%%%%%%%%%%%%%%%%%%%%%%%%%%%%%%%%%
\subsubsection{Debug}

The nodes were designed to function the same way if the microcontroller was
replaced by a 44-pin version. This leaves one port unused on the ATMEGA64A3,
which was thus used as a debug port. Five pins are available on the right side
of the device to display status on an LED or to send debug data through USARTF0.

%%%%%%%%%%%%%%%%%%%%%%%%%%%%%%%%%%%%%%%%%%%%%%%%%%%%%%%%%%%%%%%%%%%%%%%%%%%%%%%
\subsubsection{Discarded features}

Several features originally included in the design or enumerated during the
design phase were discarded. I will go over some of these and explain the
reasons behind these choices.

% TODO : discuss board layout?

%%%%%%%%%%%%%%%%%%%%%%%%%%%%%%%%%%%%%%%%%%%%%%%%%%%%%%%%%%%%%%%%%%%%%%%%%%%%%%%
\subsection{Sensor boards}

The main feature of the base board is its ``expansion port''. In order for the
nodes to share as much hardware as possible, they were all designed with the
same basic peripherals. All the additional sensors and interfaces specific to
different types of nodes were to be added on an expansion board which could be
exchanged between devices. As such, it is possible to change the structure of a
network simply by replacing a very small part of the hardware. As the plug-in
boards are very small, they are also very cheap and easily replaceable.

Although none of the current sensor boards uses all the features of the
expansion port, is was designed to provide any functionality that one may
require. The expansion port uses a two-way, 20-pin connector with a \unit[2]{mm}
pitch.

\begin{figure}[hp]
  \begin{center}
    \includegraphics[width=0.7\textwidth]{blocks/wm_sensor_io}
  \end{center}
  \caption{Sensor expansion port}
  \label{fig:sensor-io}
\end{figure}

As can be seen on the schematic, the expansion port provides the following
connectivity:
\begin{description}
  \item [Switchable power supply] The port provides two supply voltages, one of
    which can be turned on and off from the microcontroller
  \item[ADC] Several pins are connected to the analog to digital converter.
  \item[DAC] Both digital to analog pins are available.
  \item[AC] Two pins can be routed to the analog comparator module.
  \item[USART] The USART module provides serial communication.
  \item[\IIC{}] An \IIC{} bus is available to communicate with sensors.
  \item[SPI] data lines for master or slave functionality.
\end{description}

Due to the layout of the board, the sensor add-ons have to be very small. In
order to reduce the vertical space needed by the sensors nodes, the batteries
and radio transmitters were placed on the same face of the board as the add-on
connector. Therefore, the space left is only approximately \unit[25x25]{mm$^2$}.
% todo : find exact measurements
This is sufficient however for most basic applications. If more space is needed,
it is assumed that the designer will route the necessary connections on the
add-on board, then connect any other sensors off-board using connector cables.


%%%%%%%%%%%%%%%%%%%%%%%%%%%%%%%%%%%%%%%%%%%%%%%%%%%%%%%%%%%%%%%%%%%%%%%%%%%%%%%
\subsection{Gateway device}

Each network is normally comprised of one gateway device. This device uses the
same basic hardware as the sensor node, although it requires less components.

% TODO : insert picture of gateway node

Its sole purpose is to forward incoming wireless transmissions onto the
Powerline network to control the Powerline devices. In its current
implementation, it simply parses the ZigBee packets, strips them of all the
additional information (headers, checksums, addressing etc.) and then forwards
the data onto the Powerline nodes.

On one hand, this means that the nodes are free to send any data they wish,
since it does not have to be interpreted by the gateway, but at the same time,
the commands for the powerline network must be embedded into the end nodes. The
main advantage of this situation is that if some functionality is added to a
powerline module, and a wireless node is added to complement it, there is no
need to reprogram the gateway to take these new features into account.

Since the gateway device is destined to be plugged in to a wall socket, it will
always be active and listening for transmissions. It functions as the network
coordinator. Therefore, any packets which must be transmitted to the Powerline
network are to be addressed to the ZigBee coordinator, whose 64-bit address is :
\bash{00:00:00:00:00:00:00:00}.

%%%%%%%%%%%%%%%%%%%%%%%%%%%%%%%%%%%%%%%%%%%%%%%%%%%%%%%%%%%%%%%%%%%%%%%%%%%%%%%
\subsubsection{Motion sensor}

As its name indicates, the purpose of this board is to detect movement. This
task is achieved by some commercially available PIR sensors. These are cheap and
easy to use, but the downside is that they require a \unit[5]{V} supply voltage
to function. To test these functions, a mini PIR module was acquired from Seeed
Studio\furl{http://www.seeedstudio.com/depot/tiny-pir-motion-sensor-module-p-277.html}.

The specifications on this device are very terse, but it is claimed to use a
constant \unit[50]{$\mu$A} current.
% todo : give some results for power consumption of motion detector with supply

The hardware design of this board is straightforward : a charge pump converter
provides the supply voltage to the sensor(s), whose inputs are connected to two
microcontroller input pins, where they can trigger interrupts. One sensor board
is equipped with two connectors, to potentially drive two seperate PIR modules
from one wireless node.

% TODO : check name of regulator, add part number

% TODO : mention which annex contains motion sensor schematic
\begin{figure}[H]
  \begin{center}
    \includegraphics[width=0.6\textwidth]{flowcharts/motion-flowchart-colour}
  \end{center}
  \caption{Motion sensor node operation}
  \label{fig:motion-flowchart}
\end{figure}

Although they require very little power to function, the PIR modules need a
permanent supply voltage in order to function. Therefore, the power supply must
be constantly enabled. Fortunately, since these modules produce a logical
output, the radio and microcontroller can remain in deep sleep mode until the
interrupt is triggered. For optimal performance though, it would be best to use
a sensor which can operate at the same voltages as the base module, therefore
improving the overall efficiency by removing the charge pump regulator.

% TODO : insert measurements of power consumption

In the current demonstration, the motion sensor node is configured to turn on a
lamp when motion is detected. If no movement is detected for a period of 30
seconds, the light is turned off again.

% TODO : link to website and note article numbers.

%%%%%%%%%%%%%%%%%%%%%%%%%%%%%%%%%%%%%%%%%%%%%%%%%%%%%%%%%%%%%%%%%%%%%%%%%%%%%%%
\subsubsection{Environmental sensors}

The environment sensing board is comprised of three sensors:

%\begin{enum}
%  \item Barometer
%  \item Temperature sensor
%  \item Capacitive humidity sensor
%\end{enum}


The barometer and temperature were both chosen for their low consumption and
easy communication (both use the I$^{2}$C bus on the expansion port).

The capacitive humidity sensor was chosen for it's price, but as a consequence,
it is very inaccurate and requires calibration.

%%%%%%%%%%%%%%%%%%%%%%%%%%%%%%%%%%%%%%%%%%%%%%%%%%%%%%%%%%%%%%%%%%%%%%%%%%%%%%%
\subsubsection{Light controller}

The final module is intended to control lights. It include connections for two
momentary switches, as well as two potentiometers which can be turned on and
off. With this module, the microcontroller can be configured to remain in deep
sleep and only awake on a hardware interruption generated by the switches, or to
wake up periodically to sample the potentiometer.

The board itself includes footprints for SMD switches, for testing purposes, but
since the nodes are eventually meant to reside in cases, the buttons and
potentiometers are to be fixed to the enclosure and connected to the add-on
board using cables. This allows a little flexibility in the physical design of
the nodes.

\begin{figure}[!h]
  \begin{center}
    \includegraphics[width=0.5\textwidth]{flowcharts/dimmer-flowchart-colour}
  \end{center}
  \caption{Dimmer node operation}
  \label{fig:dimmer-flowchart}
\end{figure}

%%%%%%%%%%%%%%%%%%%%%%%%%%%%
% TODO : sampling the potentiometer

% extremities are connected between ground and a microcontroller pin. middle pin
% (find what that's called : cursor?) goes on adc input. pot can be turned on
% and off.



%%%%%%%%%%%%%%%%%%%%%%%%%%%%%%%%%%%%%%%%%%%%%%%%%%%%%%%%%%%%%%%%%%%%%%%%%%%%%%%
\section{Software considerations}

As all the nodes (gateway and end devices) share the same hardware platform, it
makes sense to share as much as possible of the code base between devices.

The original idea had been to write the same code for all the devices, who could
then automatically detect which role they were to play in the network, and
configure themselves at runtime to fulfil the necessary function. This approach,
though probably feasible in terms of code size, would have required too much
work early on in the hardware design, as well as more code than could probably
be expected for a semester project. It would in my mind be more interesting to
have a very modular system than to hard-code the functionality of various
modules, and have several code bases for different nodes.

With this in mind, the code is organised in the following way:

% todo : insert diagram of code hierarchy

All code shares a common root. The folder \bash{xmega\_libs} contain libraries
that are specific to the XMEGA microcontroller, mostly provided by Atmel.
Although many hardware-level functions and registers were used in the main
program, mainly for lack of time, the main objective would be to abstract all
the hardware-level functionality to these libraries, to ease portability to
another microcontroller platform.

Next we have \bash{project\_libs}, which contains all shared code relevant to
this semester project. In here, we have libraries for communicating with the
XBee radios, sending commands to the Powerline modules, as well as collecting
information from all types of sensor modules.

Finally, in the \bash{project\_libs} folder, the \bash{hardware.h} file contains
all the pin definitions relevant to the layout of the board. All functions
should use the definitions provided in this file rather than hard-code port
names. This way, existing code can be made to work on a different hardware
design simply by replacing the \bash{hardware.h} file.

Lastly, since each type of node behaves differently, it has its own set of
files. The \bash{motion\_sensor} folder contains the behaviour for a motion
sensing node, \bash{dimmer\_board} contains the code for a dimmer node, and so
on. Usually, the \bash{main.c} file is the only file that differs between the
various nodes.

%%%%%%%%%%%%%%%%%%%%%%%%%%%%%%%%%%%%%%%%%%%%%%%%%%%%%%%%%%%%%%%%%%%%%%%%%%%%%%%
\section{Low-power design}

%%%%%%%%%%%%%%%%%%%%%%%%%%%%%%%%%%%%%%%%%%%%%%%%%%%%%%%%%%%%%%%%%%%%%%%%%%%%%%%
\section{Cost considerations}

For the sake of ease of development, some sacrifices were made in the area of
cost. For starters, the ATXMEGA is not very well suited to extremely low-cost
designs, as it is quite a lot more expensive than some competing solutions,
especially the PIC24F series and Cortex M3. On the other hand, it overs a very
modern architecture with many features to experiment with, which allows a lot of
freedom for development and experimenting.

% todo : mention choice of xmegaA3 over A4 and stuff
Moreover,

The other main area were cost was overlooked at the benefit of development time
is the wireless interface. Indeed, the average cost of an XBee module is
approximately \$22, whereas it is possible to source modules from Microchip for
instance that cost less than half as much. However, these require a one-time
purchase of an expensive software stack (which is not included), as well as a
more complex development. Therefore, the XBee modules were chosen, for their
seemingly ``plug-and-play'' functionality. If more time could be spend on the
development, it would be reasonable to expect to spend up to three times less on
the radio transmitter modules, while keeping the same functionality and
compatibility with the existing modules.


%%%%%%%%%%%%%%%%%%%%%%%%%%%%%%%%%%%%%%%%%%%%%%%%%%%%%%%%%%%%%%%%%%%%%%%%%%%%%%%
\section{Test results}

\bibliographystyle{ieeetr}
\bibliography{biblio-semproj}

\pagebreak
\appendix

%%%%%%%%%%%%%%%%%%%%%%%%%%%%%%%%%%%%%%%%%%%%%%%%%%%%%%%%%%%%%%%%%%%%%%%%%%%%%%%
\section{Design changes}
\label{sec:design-changes}

As is often the case with a first prototype, some mistakes were made in the
design of the first batch of PCBs. I will not list every single modification
that has been made since the first version, but some of these bear mentioning,
in case the files or PCBs are to be re-used in a future design.

%%%%%%%%%%%%%%%%%%%%%%%%%%%%%%%%%%%%%%%%%%%%%%%%%%%%%%%%%%%%%%%%%%%%%%%%%%%%%%%
\subsection{UART Interface}
% TODO : check that it's been fixed
On the current fabricated version of the modules, the RX and TX pins were
inverted in the schematic. Obviously, the RX pin of the gateway should go to the
TX pin of the powerline module and vice versa. On the current gateway, this has
been corrected by cutting traces on the PCB. It has \emph{not} been done on the
other modules. Please ensure that you cross those two traces if you wish to use
any of the other modules to connect to a powerline module!

In addition to this, the connector footprint is upside down. Since it is a
side-facing connector, it is not possible to reverse it. Therefore, \emph{the
cable used to connect the gateway to the powerline modules must be reversed}.
This error has \emph{not been corrected} in the PCB files.

%%%%%%%%%%%%%%%%%%%%%%%%%%%%%%%%%%%%%%%%%%%%%%%%%%%%%%%%%%%%%%%%%%%%%%%%%%%%%%%
\subsection{Ambient light sensor}
The phototransistor used was chosen for its low current. However, it does not
have a very wide dynamic range. The inline resistor may need to be changed to
get better results, depending if the target application is to measure sunlight
or ambient light in a room.

%%%%%%%%%%%%%%%%%%%%%%%%%%%%%%%%%%%%%%%%%%%%%%%%%%%%%%%%%%%%%%%%%%%%%%%%%%%%%%%
\subsection{Sensor I/O}
The sensor expansion port was originally designed as on the schematic drawing,
to be reversible without causing any damage by shorting power pins to ground.
Due to an unfortunate difference in numbering between the schematic and the PCB
footprint, the pins do not appear to be in the same order as on the schematic.
Therefore please refer to the PCB drawing for correct pin names and numbers, and
take care when connecting an extension board not to reverse it.

%%%%%%%%%%%%%%%%%%%%%%%%%%%%%%%%%%%%%%%%%%%%%%%%%%%%%%%%%%%%%%%%%%%%%%%%%%%%%%%
\subsection{JTAG connector}
In order to save precious space on the circuit board, the original programming
headers were replaced with a \unit[1.27]{mm} 2-row connector for JTAG debugging
and programming. It is possible to connect any AVR compatible programmer using
either the standard \unit[2.54]{mm} JTAG or PDI connectors, using the included
adapters.

After testing, it does appear that \unit[1.27]{mm} seem a lot less robust and
less pleasant to use than the larger version. If space permits, I would
recommend using a larger connecter on a possible future revision of the board.

In addition to this, the connector is currently placed slightly too close to the
wireless transmitter. It is therefore necessary to remove the transmitter to be
able to connect or disconnect the JTAG adapter, which is obviously not very
desirable.

%%%%%%%%%%%%%%%%%%%%%%%%%%%%%%%%%%%%%%%%%%%%%%%%%%%%%%%%%%%%%%%%%%%%%%%%%%%%%%%
\subsection{JST Connector}
The JST connector included on the board was not intended to be used in the scope
of this project, and the interface has thus not been tested. On the current
version, the holes in the PCB are slightly too small. The connector pins must
therefore be filed off (with a Dremel tool for instance) before they will fit in
the footprint. This error has been corrected in the latest version of the Altium
files. % TODO : verify this

%%%%%%%%%%%%%%%%%%%%%%%%%%%%%%%%%%%%%%%%%%%%%%%%%%%%%%%%%%%%%%%%%%%%%%%%%%%%%%%
\subsection{Add-on boards}
Most features of the add-on boards have been tested and are functional, however
a few issues remain.

%%%%%%%%%%%%%%%%%%%%%%%%%%%%%%%%%%%%%%%%%%%%%%%%%%%%%%%%%%%%%%%%%%%%%%%%%%%%%%%
\subsubsection{Humidity Sensor}
\label{sub2:humidity}
As the design of the sensor boards occured fairly late in the project timeline,
some features were not extensively tested due to lack of time. Specifically, the
humidity sensor was not given much attention. The sensor was chosen for its very
low price, but the downside of this is that it requires proper calibration, as
the absolute humidity rating can vary by over \unit[40\%]{RH} (on a scale of
zero to \unit[100\%]{RH}, obviously). The sensor is rather useless without a
proper calibration process, which requires either an environment with controlled
humidity, or a well-calibrated reference sensor. At the time of writing, it has
not yet been tested, and therefore the charging resistor may need to be
recalculated and modified.

% Barometer
%%%%%%%%%%%%%%%%%%%%%%%%%%%%%%%%%%%%%%%%%%%%%%%%%%%%%%%%%%%%%%%%%%%%%%%%%%%%%%%
\subsubsection{Barometer}
\label{sub2:barometer}
The barometer chip on the environmental sensor board has been implemented and
successfully tested. Raw ADC readings for pressure and temperature can be made,
as well as loading of the calibration coefficients. An attempt was made
at porting the example code from the Freescale application
note\cite{freescale3785}

It seems however that the results are not quite accurate. It was deemed
therefore that the time needed to debug this would be better spent elsewhere, as
some other areas are in need of more attention.

\pagebreak
%%%%%%%%%%%%%%%%%%%%%%%%%%%%%%%%%%%%%%%%%%%%%%%%%%%%%%%%%%%%%%%%%%%%%%%%%%%%%%%
\section{Schematics}
% todo : include schematics

\subsection{Wireless base module}
\includepdf[landscape=true, pages={1,2,3}]{pdfs/WirelessModule_Schematics.PDF}

\subsection{Motion sensor add-on}
\includepdf[landscape=true, pages={1,2}]{pdfs/motionboard}

\subsection{Dimmer add-on}
\includepdf[landscape=true, pages={1,2}]{pdfs/dimmerboard}

\subsection{Environmental sensor add-on}
\includepdf[landscape=true, pages={1,2}]{pdfs/sensorsboard}

\pagebreak
%%%%%%%%%%%%%%%%%%%%%%%%%%%%%%%%%%%%%%%%%%%%%%%%%%%%%%%%%%%%%%%%%%%%%%%%%%%%%%%
\section{Microcontroller software}
The following section contains the source code for the nodes. The XMEGA
libraries by Atmel are available %TODO : here

% TODO : link to github repo
The source code for the project is versioned using git and hosted on
Github\furl{http://github.com}. For the foreseeable future, the code will live
at the following address : \url{http://github.com/tunebird/semester-1-code}.


%%%%%%%%%%%%%%%%%%%%%%%%%%%%%%%%%%%%%%%%%%%%%%%%%%%%%%%%%%%%%%%%%%%%%%%%%%%%%%%
\subsection{Project Libraries}
\label{an:project-libs}

%%%%%%%%%%%%%%%%%%%%%%%%%%%%%%%%%%%%%%%%%%%%%%%%%%%%%%%%%%%%%%%%%%%%%%%%%%%%%%%
\subsubsection{XBee code}
\bash{xbee.h}
\lstinputlisting[language=C]{../code/project_libs/xbee.h}
\bash{xbee.c}
\lstinputlisting[language=C]{../code/project_libs/xbee.c}

%%%%%%%%%%%%%%%%%%%%%%%%%%%%%%%%%%%%%%%%%%%%%%%%%%%%%%%%%%%%%%%%%%%%%%%%%%%%%%%
\subsubsection{Hardware definitions}
\bash{hardware.h}
\lstinputlisting[language=C]{../code/project_libs/hardware.h}

%%%%%%%%%%%%%%%%%%%%%%%%%%%%%%%%%%%%%%%%%%%%%%%%%%%%%%%%%%%%%%%%%%%%%%%%%%%%%%%
\subsubsection{Debug functions}
\bash{debug.h}
\lstinputlisting[language=C]{../code/project_libs/debug.h}
\bash{debug.c}
\lstinputlisting[language=C]{../code/project_libs/debug.c}

%%%%%%%%%%%%%%%%%%%%%%%%%%%%%%%%%%%%%%%%%%%%%%%%%%%%%%%%%%%%%%%%%%%%%%%%%%%%%%%
\subsubsection{Powerline interface}
\bash{powerline.h}
\lstinputlisting[language=C]{../code/project_libs/powerline.h}
\bash{powerline.c}
\lstinputlisting[language=C]{../code/project_libs/powerline.c}

%%%%%%%%%%%%%%%%%%%%%%%%%%%%%%%%%%%%%%%%%%%%%%%%%%%%%%%%%%%%%%%%%%%%%%%%%%%%%%%
\subsubsection{Built-in sensors}
\bash{builtin\_sensors.h}
\lstinputlisting[language=C]{../code/project_libs/builtin_sensors.h}
\bash{builtin\_sensors.c}
\lstinputlisting[language=C]{../code/project_libs/builtin_sensors.c}

%%%%%%%%%%
\pagebreak
%%%%%%%%%%


%%%%%%%%%%%%%%%%%%%%%%%%%%%%%%%%%%%%%%%%%%%%%%%%%%%%%%%%%%%%%%%%%%%%%%%%%%%%%%%
\subsection{Motion Sensor Node}
\label{an:motion-code}

%%%%%%%%%%%%%%%%%%%%%%%%%%%%%%%%%%%%%%%%%%%%%%%%%%%%%%%%%%%%%%%%%%%%%%%%%%%%%%%
\subsubsection{Motion sense library}
\bash{board\_motion.h}
\lstinputlisting[language=C]{../code/project_libs/board_motion.h}
\bash{board\_motion.c}
\lstinputlisting[language=C]{../code/project_libs/board_motion.c}

%%%%%%%%%%%%%%%%%%%%%%%%%%%%%%%%%%%%%%%%%%%%%%%%%%%%%%%%%%%%%%%%%%%%%%%%%%%%%%%
\subsubsection{Motion sense main file}
\bash{main.c}
\lstinputlisting[language=C]{../code/node-motion/main.c}

%%%%%%%%%%
\pagebreak
%%%%%%%%%%

%%%%%%%%%%%%%%%%%%%%%%%%%%%%%%%%%%%%%%%%%%%%%%%%%%%%%%%%%%%%%%%%%%%%%%%%%%%%%%%
\subsection{Dimmer Node}
\label{an:dimmer-code}

%%%%%%%%%%%%%%%%%%%%%%%%%%%%%%%%%%%%%%%%%%%%%%%%%%%%%%%%%%%%%%%%%%%%%%%%%%%%%%%
\subsubsection{Dimmer library}
\bash{board\_dimmer.h}
\lstinputlisting[language=C]{../code/project_libs/board_dimmer.h}
\bash{board\_dimmer.c}
\lstinputlisting[language=C]{../code/project_libs/board_dimmer.c}

%%%%%%%%%%%%%%%%%%%%%%%%%%%%%%%%%%%%%%%%%%%%%%%%%%%%%%%%%%%%%%%%%%%%%%%%%%%%%%%
\subsubsection{Dimmer main file}
\bash{main.c}
\lstinputlisting[language=C]{../code/node-dimmer/main.c}

%%%%%%%%%%
\pagebreak
%%%%%%%%%%

%%%%%%%%%%%%%%%%%%%%%%%%%%%%%%%%%%%%%%%%%%%%%%%%%%%%%%%%%%%%%%%%%%%%%%%%%%%%%%%
\subsection{Environment Sensing Node}
\label{an:environment-code}

%%%%%%%%%%%%%%%%%%%%%%%%%%%%%%%%%%%%%%%%%%%%%%%%%%%%%%%%%%%%%%%%%%%%%%%%%%%%%%%
\subsubsection{Environment sense library}
\bash{board\_sensors.h}
\lstinputlisting[language=C]{../code/project_libs/board_sensors.h}
\bash{board\_sensors.c}
\lstinputlisting[language=C]{../code/project_libs/board_sensors.c}

%%%%%%%%%%%%%%%%%%%%%%%%%%%%%%%%%%%%%%%%%%%%%%%%%%%%%%%%%%%%%%%%%%%%%%%%%%%%%%%
\subsubsection{Environment sense main file}
\bash{main.c}
\lstinputlisting[language=C]{../code/node-sensors/main.c}

%%%%%%%%%%
\pagebreak
%%%%%%%%%%

%%%%%%%%%%%%%%%%%%%%%%%%%%%%%%%%%%%%%%%%%%%%%%%%%%%%%%%%%%%%%%%%%%%%%%%%%%%%%%%
\subsection{ZigBee to Powerline Gateway}
\label{an:gateway-code}

%%%%%%%%%%%%%%%%%%%%%%%%%%%%%%%%%%%%%%%%%%%%%%%%%%%%%%%%%%%%%%%%%%%%%%%%%%%%%%%
\subsubsection{Environment sense main file}
\bash{gateway.c}
\lstinputlisting[language=C]{../code/gateway/gateway.c}

%%%%%%%%%%
\pagebreak
%%%%%%%%%%


\end{document}

